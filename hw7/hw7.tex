\documentclass[11pt]{article}

\setlength\topmargin{-0.6cm}   
\setlength\textheight{23.4cm}
\setlength\textwidth{17.0cm}
\setlength\oddsidemargin{0cm} 
 
\usepackage{hyperref}
\urlstyle{same}


\begin{document}

\begin{center}
\LARGE
LING572 HW7: Math needed for Neural Networks\\
Due: 11pm on Feb 22, 2022\\
\vspace{0.3in}
\end{center}

A few notes about this assignment:
\begin{itemize}

 \item The answers to the questions should be pretty short. I've left some space for you to fill in the answers. I've also made the \LaTeX \ file available in case you want to add the answers to the latex file directly. In that case, you need to run pdf2latex, latexmk, or something like that to generate a pdf from the \LaTeX \ file.

 \item If you prefer to write formulas on paper (instead of typing them with \LaTeX \ or Word), it's ok. You just need to fill out the rest of the assignment, print out the file, insert formulas by hand, scan the paper, and then submit via Canvas.

 \item Since no programming is required, you only need to submit a single
     file. Please call it {\bf readme.pdf}.

  \item The assignment has three parts:
    \begin{itemize}
    \item Q1-Q2 are on the derivative of a univariate function
      (a function with a single variable),
      which should be covered in college-level calculus (a prerequisite
      of LING572).

    \item Q3 is on the partial derivates of a multivariate function.
      If you have not learned that topic before,
      you can look at the tutorials provided in Q3.

   
    \item Q4 is on softmax, which might be new to you. I include
      the url of a short tutorial on the function.
      
    \end{itemize}


  \item There are tons of textbooks and online tutorials that cover
    those topics. If the links provided in Q3-Q4 do not work for you
    or you are still confused after going over them, feel free to
    read any calculus textbook or search the Internet for more info.

\end{itemize}




%%%%%%%%%%%%%%%%%%%
\vspace{0.4 in}
\noindent
       {\bf Q1 (12 points):} Let $f'(x)$ denote the derivative of a
         univariate function $f(x)$ w.r.t. the variable $x$.
\begin{description}
   \item [(a) 2 pts:] What does f'(x) intend to measure? \\ \\
  
   \item [(b) 2 pts:] Let $h(x)=f(g(x))$. What is $h'(x)$ in terms of
       f'(x) and g'(x)? \\ \\

   \item [(c) 2 pts:] Let $h(x)=f(x)g(x)$. What is $h'(x)$? \\ \\

   \item [(d) 3 pts:] Let $f(x)=a^x$, where $a>0$. What is $f'(x)$? \\ \\

   \item [(e) 3 pts:] Let $f(x)= x^{10}-2x^8 + \frac{4}{x^2} + 10$.
            What is $f'(x)$? \\ \\ \\
\end{description}


%%%%%%%%%%%%%%%%%%%
\vspace{0.4 in}
\noindent
{\bf Q2 (18 points):} The logistic function is $f(x)=\frac{1}{1+e^{-x}}$.
       The tanh function is $g(x)=\frac{e^x - e^{-x}}{e^x +e^{-x}}$.
    \begin{description}
     \item [(a) 6 pts:] Prove that $f'(x)=f(x)(1-f(x))$. \\ \\ \\ \\ \\
     \item [(b) 6 pts:] Prove that $g'(x)=1 - g^2(x)$. \\  \\ \\ \\   \\
     \item [(c) 6 pts:] Prove that $g(x) = 2f(2x)-1$  \\  \\ \\ \\ \\ \\ \\ \\
    \end{description}
    


%%%%%%%%%%%%%%%%%%%
\vspace{0.4 in}
\noindent
    {\bf Q3 (45 points):} Let $f$ be a multi-variate function, and let $x$
    be one of the variables in $f$. Let us denote the partial derivative of
    $f$ with respect to $x$ by $f'_x$ or
    $\frac{df}{dx}$ or $\frac{\partial f}{\partial x}$.
    Please answer the following questions:

\begin{description}

\item [(a) 15 free pts:] Refresh your memory about gradient, partial derivative,
  chain rule. Here are some readings on this. Free free to skip them
  if you already know the content. On the other hand, if you need more
  info or cannot access the videos as youtube is blocked in your country,
  just search for ``partial derivatives'',
  ``gradient'', and ``chain rule with partial derivatives''.
  There should be tons of materials on those topics on the
  Internet. 
  \begin{itemize}
  \item Khan Academy's page on partial derivatives: \\
    \url{https://www.khanacademy.org/math/multivariable-calculus/multivariable-derivatives/partial-derivative-and-gradient-articles/a/introduction-to-partial-derivatives}
    
  \item Khan Academy's page on the gradient: \\
    \url{https://www.khanacademy.org/math/multivariable-calculus/multivariable-derivatives/partial-derivative-and-gradient-articles/a/the-gradient}

  \item Chain rule with partial derivatives: \\
    \url{https://tutorial.math.lamar.edu/classes/calciii/chainrule.aspx}
    
  \item A 28-min tutorial on partial derivatives and the gradient: \\
     \url{https://www.youtube.com/watch?v=CnVes9TdnPo&ab_channel=TheOrganicChemistryTutor}

  \item A 21-min video on Chain rule with partial derivatives: \\
     \url{https://www.youtube.com/watch?v=XipB_uEexF0&ab_channel=TheOrganicChemistryTutor}
     
  \item A one-hour tutorial on Partial derivatives: \\
    \url{https://www.youtube.com/watch?v=JAf_aSIJryg&ab_channel=TheOrganicChemistryTutor} \\ \\

  \end{itemize}

  
\item [(b) 3 pts:] What is the partial derivative $f'_x$ trying to measure?
     \\ \\ \\ \\


\item [(c) 3 pts:] How do you calculate the gradient of $f$ at a point $z$? \\ \\ \\ \\

\item [(d) 5 pts:]  Suppose that $x=g(t)$  and $y=h(t)$ are differentiable
  functions of $t$ and $z=f(x,y)$ is a differentiable function of
  $x$ and $y$. How do you calculate $\frac{\partial z}{\partial t}$ using the chain rule
  of partial derivatives?\\ \\ \\ \\

  
\item [(e) 6 pts:] Let $f(x,y)=x^3 + 3x^2y+y^3 + 2x$. \\
  
  What is $f'_x$?
  What is $f'_y$? \\ \\  \\ \\


  What is the gradient of $f(x, y)$ at point (1, 2)? \\ \\ \\ \\ 

  
  
\item [(f) 3 pts:] Let $z = \sum_{i=1}^n w_i x_i$.
  What is $\frac{\partial z}{\partial w_i}$? \\ \\ \\ \\
  
\item [(g) 5 pts:] Let $f(z)=\frac{1}{1+e^{-z}}$ and $z = \sum_{i=1}^n w_i x_i$. \\
       What is $\frac{\partial f}{\partial z}$? \\ \\ \\ \\  
       What is $\frac{\partial f}{\partial w_i}$? \\ \\ \\ \\

       
       Hint: Use chain rule and your answers should contain $f(z)$. \\ \\

\item [(h) 5 pts:] Let $E(z)=\frac{1}{2}(t - f(z))^2$,
  $f(z)=\frac{1}{1+e^{-z}}$ and $z = \sum_{i=1}^n w_i x_i$. \\
  
  What is $\frac{\partial E}{\partial w_i}$? Hint: the answer should contain $f(z)$. \\ \\ \\ \\ \\ 
\end{description}


  
%%%%%%%%%%%%%%%%%%%
\vspace{0.4 in}
\noindent
{\bf Q4 (25 points):} The softmax function: please read the short tutorial at \\
\url{https://deepai.org/machine-learning-glossary-and-terms/softmax-layer} \\
and answer the following questions:
    
\begin{description}
\item [(a) 2 pts:] The softmax function is a function that takes the input
  {\it x} and produces the output {\it y}. What is the type of x?
  What is the type of y? \\ \\ \\ \\
  
\item [(b) 5 pts:] In general which layer in neural network (NN) is the softmax function used and why? \\ \\ \\ \\

\item [(c) 5 pts:] What is the relationship between the softmax function
  and the sigmoid function?  \\ \\ \\ \\

\item [(d) 7 pts:] What is the relationship between the softmax function
  and the argmax function? In NN, when do you use softmax and
  when do you use argmax? \\ \\ \\ \\

\item [(e) 6 pts:] If a vector x is [1, 2, 3, -1, -4, 0], what is
  softmax(x)? What is argmax(x)? \\ \\ \\ \\
  
  
\end{description}
 






%%%%%%%%%%%%%%%%%%
\vspace{0.5 in}
\noindent
{\bf Submission:}  Submit the following to Canvas:

\begin{itemize}
\item Since HW7 has no coding part, you only need to submit your {\bf readme.pdf}
  which includes answers to all the questions, plus anything you want
  TA to know. No need to submit anything else. 
  
\end{itemize}

\end{document}
