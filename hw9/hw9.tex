\documentclass[11pt]{article}

\setlength\topmargin{-0.6cm}   
\setlength\textheight{23.4cm}
\setlength\textwidth{17.0cm}
\setlength\oddsidemargin{0cm} 

\usepackage{hyperref}

\begin{document}

\begin{center}
\LARGE
HW9: Feedforward NN\\
Due: 11pm on March 8, 2022\\
\vspace{0.3in}
\end{center}


The assignment has two parts:
\begin{itemize}
    \item Q1-Q2: Please read chapter 1 of NN book at 
      \url{http://neuralnetworksanddeeplearning.com/chap1.html}
      and answer the questions. You can also check the hw9 code
      to get a more concrete idea of how feedforward NN works.
      
    \item Q3-Q5: Check an implementation of feedforward NN,
      and test its performance with respect to different hyperparameters
      and activation functions. Add a few lines of code to
      implement tanh as an activation function. \\ \\
\end{itemize}


The hw9 code is under dropbox/21-22/572/hw9/code/:
\begin{itemize}
\item hw9.sh: the shell script that calls hw9\_script.py.
  \begin{itemize}
    \item To run hw9.sh, the command line is ``./hw9.sh config\_file output\_file''
    \item config\_file specifies the hyperparameters of the NN
      (e.g., {\bf config1.yml})
      
    \item output\_file will be a file consisting of ``\# correct / \# instances'' per epoch. For the test accuracy in Table 1 and 2,
      just run hw9.sh with a config\_file and
      report the final line as a percentage.
      
    \item The script will print runtime to stdout. 
  \end{itemize}
  
  \item hw9\_script.py: the python script that reads in the training data and test set, and call network.py to do training and testing.

  \item network.py: the code that defines the architecture of the NN,
         including the activation function and its derivative. 
\end{itemize}

    
%%%%%%%%%%%%%%%%%%%
\vspace{0.4in}
\noindent
{\bf Q1 (15 points):} Suppose a feedforward NN
  (sometimes called multilayer perceptron or MLP) has
  $m$ layers: the input layer is the 1st layer, the output layer
  is the last layer, and there are $m-2$ hidden layers in between. 
  The number of neurons in the $k^{th}$ layer is $n_k$.
  Each neuron in one layer is connected to
  every neuron in the next layer and there are no other connections.
  Let's ignore the bias for Q1. That is, the model only has weight $w$,
  and there is no bias $b$.
\begin{description}

\item [(a) 5 pts:] How many model parameters (i.e., weights) are there in
                   this network? \\ \\

\item [(b) 10 pts:] Let $x$ be a column vector\footnote{A row vector
            is a $1 \times n$ matrix
          (e.g., $[a_1, a_2, ..., a_n]$); a column vector is
          a $n \times 1$ matrix. If you transpose a row vector, you get a
          column vector. For more info, see \url{https://en.wikipedia.org/wiki/Row_and_column_vectors}}
         %
         that denotes the values of the
  input layer. Let $M_k$ denote the weight matrix
  between layer $k$ and $k+1$; that is, the cell $a_{i,j}$
  in $M_k$ stores
  the weight on the arc from the $j^{th}$ neuron in layer $k$
  to the $i^{th}$ neuron in layer $k+1$.
  Let's assume all the layers use the same activation function $g$, which
  is a function from R to R (R is the set of real numbers).
  If $x$ is a column vector, $g(x)$
  will apply the function $g$ to every element in $x$ and return a column
  vector as the output.
  
  \begin{itemize}
  \item Given the input $x$ ($x$ is a column vector),
      what is the formula for calculating
      the output of the first hidden layer?  \\ \\

    \item Given the input $x$, what is the formula for calculating
      the output of the output layer? \\ \\ \\
  \end{itemize}
  

\item [Hint:] In class, we show the formula for calculating the $z$ and $y$
  value for a neuron, where $z = b + \sum_j w_j x_j$ and $y=g(z)$.
  In Q1, let's assume $b$ is zero.
  
  Now let's look at layer $i$: the outputs of the neurons in that layer form
  a column vector of the size $n_i \times 1$. Let's call this vector $y_i$.
  The question is how one can calculate the output vector, $y_{i+1}$, at layer
  $i+1$ given the weight matrix $M_k$ and $y_k$. Once you figure out how
  to do that, you can apply the same formula layer by layer, in order
  to get the output of the output layer given the input $x$.

\end{description}  



%%%%%%%%%%%%%%%%%%%%%%
\vspace{0.4in}
\noindent
{\bf Q2 (30 points):} Suppose that you are training a neural network to do text classification, with $n > 2$ classes.

\begin{description}
\item [(a) 5 pts:] What loss function can you use for the output layer?
  Can the loss function be the error rate on the training data? Why or why not?
  Here, let's assume that you need to use backpropagation for training.
  

\item [(b) 15 pts:] What are the main idea and benefit
  of stochastic gradient descent? \\ \\

  What is a training epoch? \\ \\

  Let $T$ be the size of the training data (i.e., the number of training
  instances), $m$ be the size of mini-batch,
  and your training process contains $E$ training epoches.
  How many times is each weight in the NN updated? \\ \\ 


\item [(c) 10 pts:] How can one choose the learning rate? What's the risk
  if the rate is too big? What's the risk if the rate is too
  small? \\ \\ 

\end{description}


%%%%%%%%%%%%%%%%%%%%%
\vspace{0.5 in}
\hspace{-0.3in}
 {\bf Q3 (15 points):} Run hw9.sh with different config file settings
 (e.g., {\bf config1.yml} and {\bf config2.yml} are the config files
 for the first two experiments in Table 1).
 The {\bf activation} value in the config file should be set to 0 (for sigmoid function). For the learning rate, keep it as 0.5. Fill out Table 1.
         

\begin{table}[h]
\centering
\caption{Classification accuracy with {\bf sigmoid} activation function}
\label{table1}
  \begin{tabular}{|c|r|l|l|l|l|l|} \hline
    Expt & \# of          & \# of neurons in & \# of    & mini-batch & test  & CPU time   \\  
    id   &  hidden layer  & hidden layers    & epoches & size       & accuracy  & (in minutes) \\ \hline

    1   &  1              &  30              &  30     &  10       &     & \\ \hline
    2   &  1              &  30              &  30     &  50       &     & \\ \hline
    3   &  1              &  30              &  100    &  10       &     &  \\ \hline
    4   &  1              &  60              &  30     &  10       &     & \\ \hline
    5   &  2              &  30, 30          &  30     &  10       &     & \\ \hline

    6   &  2              &  40, 20          &  30     &  10       &     & \\ \hline
    
    7   &  3              &  20, 20, 20      &  30     &  10       &     & \\ \hline
  \end{tabular}
\end{table}


%%%%%%%%%%%%%%%%%%
\vspace{0.5 in}
\hspace{-0.3in}
{\bf Q4 (20 points):} Modify one or more python files in the hw9/code/ directory so that the new code will use tanh when {\bf activation} value in the config file is set to 1. For the learning rate, keep it as 0.5.
\begin{itemize}
  \item Fill out Table 2, which is the same as Table 1, except that it uses {\bf tanh} as the activation function.
  
  \item In the readme.[txt $\mid$ pdf], explain which functions (or which lines) in which file(s) you have changed.
    
  \item Submit the modified python code. Please keep the file names unchanged.
\end{itemize}


\begin{table}[h]
\centering
\caption{Classification accuracy with {\bf tanh} activation function}
\label{table1}
  \begin{tabular}{|c|r|l|l|l|l|l|} \hline
    Expt & \# of          & \# of neurons in & \# of    & mini-batch & test  & CPU time   \\  
    id   &  hidden layer  & hidden layers    & epoches & size       & accuracy  & (in minutes) \\ \hline

    1   &  1              &  30              &  30     &  10       &     & \\ \hline
    2   &  1              &  30              &  30     &  50       &     & \\ \hline
    3   &  1              &  30              &  100    &  10       &     &  \\ \hline
    4   &  1              &  60              &  30     &  10       &     & \\ \hline
    5   &  2              &  30, 30          &  30     &  10       &     & \\ \hline

    6   &  2              &  40, 20          &  30     &  10       &     & \\ \hline
    
    7   &  3              &  20, 20, 20      &  30     &  10       &     & \\ \hline
  \end{tabular}
\end{table}


           
%%%%%%%%%%%%%%%%%%
\vspace{0.5 in}
\hspace{-0.3in}
{\bf Q5 (20 points):} Answer the following questions:

\begin{description}
\item [(a) 10 pts:] For any experiment in Q3 or Q4, if you run it multiple
  times, you are likely to get different results. Why is that?
  \begin{itemize}
  \item specify the line numbers in the python code that causes
        this non-deterministic behavior of the system. 
  \item Is this kind of behavior desireable? Why or why not?  
  \end{itemize}

\item [(b) 10 pts:] What can you conclude from Tables 1 and 2, and from
        Q5(a)?

\end{description}
           





%%%%%%%%%%%%%%%%%%
\vspace{0.5 in}
\noindent
{\bf Submission:}  Submit the following to Canvas:

\begin{itemize}
  \item Your note file {\it readme.(txt $\mid$ pdf)}, which
     includes your answers to Q1, Q2, Q4, Q5, and Tables 1-2, 
     and any notes that you want the TA to read.
      

  \item  hw.tar.gz that includes two python files specified in
      dropbox/21-22/572/hw9/submit-file-list.

  \item Make sure that you run {\bf check\_hw9.sh} before
    submitting your hw.tar.gz.
    
\end{itemize}

\end{document}
